\documentclass[addtable,answer,twoside,12pt]{hnuexam}
\shadedsolutions
\definecolor{SolutionColor}{rgb}{0.8,0.9,1}
%\framedsolutions
%\unframedsolutions

\begin{document}
\examinformation%
{衡阳师范学院2018-2019 学年第二学期}%
{化学与材料科学学院化学专业2020级}%
{《高等数学(\Rmnum{2})》期末考试试题A卷}%
{闭卷}%
{120}%

\vspace{-1.2em}

\sectiongradetable

\begin{questions}
	\makepart{选择题}[3]{15}
	\question
	\num{.3e45}\si{\newton} = \hfill $(\qquad)$

	\begin{oneparchoices}
		\choice \num{.3e45}
		\choice	\ang{12.3}
		\CorrectChoice \num{.3e45}\si{kg.m/s^2}
		\choice \num{3e45}\si{\kilo\gram\metre\per\square\second}
	\end{oneparchoices}

	\question
		求初值问题$y^\prime=y,y(0)=1$的特解为$y=$\hfill $(\qquad)$
	
		\begin{oneparchoices}
			\choice $e^x+1$
			\choice	$\frac{1}{2}x^2+1$
			\choice $x^2+C,\text{其中}C$为任意常数
			\CorrectChoice $e^x$
		\end{oneparchoices}

\question
	求初值问题$y^\prime=y,y(0)=1$的特解为$y=$\hfill $(\qquad)$

	\begin{oneparchoices}
		\choice $e^x+1$
		\choice	$\frac{1}{2}x^2+1$
		\choice $x^2+C,\text{其中}C$为任意常数
		\CorrectChoice $e^x$
	\end{oneparchoices}


\question
	求初值问题$y^\prime=y,y(0)=1$的特解为$y=$\hfill $(\qquad)$

	\begin{oneparchoices}
		\choice $e^x+1$
		\choice	$\frac{1}{2}x^2+1$
		\choice $x^2+C,\text{其中}C$为任意常数
		\CorrectChoice $e^x$
	\end{oneparchoices}


	\question
		求初值问题$y^\prime=y,y(0)=1$的特解为$y=$\hfill $(\qquad)$
	
		\begin{oneparchoices}
			\choice $e^x+1$
			\choice	$\frac{1}{2}x^2+1$
			\choice $x^2+C,\text{其中}C$为任意常数
			\CorrectChoice $e^x$
		\end{oneparchoices}

	\makepart{填空题}[3]{15}
	\question 求椭圆$\frac{x^2}{4}+y^2=2$在点$(-2,1)$处的切线方程\fillin[$x-2y+4=0$][1.5in].
	\question 求椭圆$\frac{x^2}{4}+y^2=2$在点$(-2,1)$处的切线方程\fillin[$x-2y+4=0$][2in].
	\question 吃饭,睡觉,\fillin[打豆豆][0.5in].	
	\question 求椭圆$\frac{x^2}{4}+y^2=2$在点$(-2,1)$处的切线方程\fillin[$x-2y+4=0$][2in].
	
	\makepart{判断题}[2]{10}
	\question 若二元函数$f(x,y)$在点$(1,1)$处连续,则其在该点处可微。\hfill\tf[\XSolidBrush]
	\question 如果常数项级数$\sum\limits_{n=1}^{\infty}a_n$收敛,那么$\lim\limits_{n\to \infty}a_n=0$.\hfill\tf[\Checkmark]
	\question 若二元函数$f(x,y)$在点$(1,1)$处连续,则其在该点处可微。\hfill\tf[\XSolidBrush]
	\question 如果常数项级数$\sum\limits_{n=1}^{\infty}a_n$收敛,那么$\lim\limits_{n\to \infty}a_n=0$.\hfill\tf[\Checkmark]
		\question 如果常数项级数$\sum\limits_{n=1}^{\infty}a_n$收敛,那么$\lim\limits_{n\to \infty}a_n=0$.\hfill\tf[\Checkmark]
		
	\makepart{解答题}{60}
	\question
	试将微分方程$x\frac{\mathrm{d}y}{\mathrm{d}x}=x^2+3y,\,x>0$转换成一阶非齐次线性微分方程的标准形式,然后使用常数变易法求解,最后对求得的结果进行验算。
	\begin{solution}
		一阶非齐次线性微分方程的标准形式为:$\frac{\mathrm{d}y}{\mathrm{d}x}+\frac{-3}{x}y=x.$\score{2}

		先解其对应的齐次线性微分方程$\frac{\mathrm{d}y}{\mathrm{d}x}+\frac{-3}{x}y=0.$得:$y=cx^3$,其中$c$为任意的实常数。\score{5}

		使用常数变易法将常数$c$替换成与$x$相关的函数$c(x)$代入原微分方程解得:$\frac{\mathrm{d}c(x)}{\mathrm{d}x}=\frac{1}{x^2}$,即$c(x)=-\frac{1}{x}+C$,其中$C$为任意常数。故原微分方程的通解为:
		\[
			y=Cx^3-x^2,x>0\quad \text{其中}C\text{为任意常数。}\score{8}
		\]
		检验:代入原微分方程,左边为$x(3Cx^2-2x)=3Cx^3-2x^2$,右边为$x^2+3Cx^3-3x^2=3Cx^3-2x^2.$验算可得结果正确。\score{10}
	\end{solution}
	\vspace*{\stretch{1}}
	\clearpage

	\question
	试求出不共线三点$P(1,-1,0),Q(2,1,-1),R(-1,1,2)$所确定的平面的单位法向量。
	\begin{solution}
		设法向量为$\vec{n}$,则$\vec{n}=\vec{PQ}\times\vec{PR}=\begin{vmatrix}
				\vec{i} & \vec{j} & \vec{k} \\
				1       & 2       & -1      \\
				-2      & 2       & 2
			\end{vmatrix}=(6,0,6).$\score{7}

		故其单位法向量为$\pm\frac{\vec{n}}{|\vec{n}|}=\pm\frac{\sqrt{2}}{2}(1,0,1).$\score{10}
	\end{solution}
	\vspace*{\stretch{1}}


\question
	试求出不共线三点$P(1,-1,0),Q(2,1,-1),R(-1,1,2)$所确定的平面的单位法向量。
	\begin{solution}
		设法向量为$\vec{n}$,则$\vec{n}=\vec{PQ}\times\vec{PR}=\begin{vmatrix}
				\vec{i} & \vec{j} & \vec{k} \\
				1       & 2       & -1      \\
				-2      & 2       & 2
			\end{vmatrix}=(6,0,6).$\score{7}

		故其单位法向量为$\pm\frac{\vec{n}}{|\vec{n}|}=\pm\frac{\sqrt{2}}{2}(1,0,1).$\score{10}
	\end{solution}
	\vspace*{\stretch{1}}

	\question
		试求出不共线三点$P(1,-1,0),Q(2,1,-1),R(-1,1,2)$所确定的平面的单位法向量。
		\begin{solution}
			设法向量为$\vec{n}$,则$\vec{n}=\vec{PQ}\times\vec{PR}=\begin{vmatrix}
					\vec{i} & \vec{j} & \vec{k} \\
					1       & 2       & -1      \\
					-2      & 2       & 2
				\end{vmatrix}=(6,0,6).$\score{7}
	
			故其单位法向量为$\pm\frac{\vec{n}}{|\vec{n}|}=\pm\frac{\sqrt{2}}{2}(1,0,1).$\score{10}
		\end{solution}
		\vspace*{\stretch{1}}
	\clearpage

	\question
	求函数$f(x,y)=x+y$在$g(x,y)=x^2+y^2=1$限制下的条件最大值与最小值。(提示:可以使用拉格朗日乘数法。)
	\begin{solution}
		注:此题也可以不使用乘数法。小题可以看几何意义,大题可以用三角函数代换。另外也可以使用从限制条件中解出$y$代入$f$来解无条件极值。

		设$L(x,y,\lambda)=f(x,y)-\lambda [g(x,y)-1]$
		由$\begin{cases}
				\frac{\partial L}{\partial x}=1-2\lambda x=0 \\
				\frac{\partial L}{\partial x}=1-2\lambda y=0 \\
				\frac{\partial L}{\partial \lambda}=x^2+y^2-1=0
			\end{cases}$\score{4}

		由前两式相减得$x=y$或者$\lambda=0$(舍去)。\score{5}

		将$x=y$代入最后一式得$x^2=\frac{1}{2}$,所以$x=y=\frac{\sqrt{2}}{2}$或者$x=y=-\frac{\sqrt{2}}{2}$.\score{7}

		于是$f$的条件极值为$f(\frac{\sqrt{2}}{2},\frac{\sqrt{2}}{2})=\sqrt{2},f(-\frac{\sqrt{2}}{2},-\frac{\sqrt{2}}{2})=-\sqrt{2}.$\score{9}

		综上所述,$f$的最大值为$\sqrt{2}$,最小值为$-\sqrt{2}.$\score{10}
	\end{solution}
	\vspace*{\stretch{1}}

	\question
		朱自清是怎么描写时间过得比较快的?
		\begin{solution}
			去的尽管去了,来的尽管来着;去来的中间,又怎样地匆匆呢?早上我起来的时候,小屋里射进两三方斜斜的太阳。太阳他有脚啊,轻轻悄悄地挪移了;我也茫茫然跟着旋转。于是——洗手的时候,日子从水盆里过去;吃饭的时候,日子从饭碗里过去;默默时,便从凝然的双眼前过去。我觉察他去的匆匆了,伸出手遮挽时,他又从遮挽着的手边过去,天黑时,我躺在床上,他便伶伶俐俐地从我身上跨过,从我脚边飞去了。等我睁开眼和太阳再见,这算又溜走了一日。我掩着面叹息。但是新来的日子的影儿又开始在叹息里闪过了。
			
 			在逃去如飞的日子里,在千门万户的世界里的我能做些什么呢?只有徘徊罢了,只有匆匆罢了;在八千多日的匆匆里,除徘徊外,又剩些什么呢?过去的日子如轻烟,被微风吹散了,如薄雾,被初阳蒸融了;我留着些什么痕迹呢?我何曾留着像游丝样的痕迹呢?我赤裸裸来到这世界,转眼间也将赤裸裸的回去罢?但不能平的,为什么偏要白白走这一遭啊?
		\end{solution}
	\vspace*{\stretch{1}}
\end{questions}
\end{document}
